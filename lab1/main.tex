\documentclass[14pt,a4paper,article]{article}
\usepackage[a4paper, mag=1000, left=2.5cm, right=1cm, top=2cm, bottom=2cm, headsep=0.7cm, footskip=1cm]{geometry}
\usepackage[utf8]{inputenc}
\usepackage[T2A]{fontenc}
\usepackage[russian]{babel} %[english,russian]
\usepackage{indentfirst}
\usepackage[dvipsnames]{xcolor}
\usepackage[colorlinks]{hyperref}
\usepackage{amsfonts}
\usepackage{amsmath}
\usepackage{listings}
\usepackage{xcolor}

\definecolor{codegreen}{rgb}{0,0.6,0}
\definecolor{codegray}{rgb}{0.5,0.5,0.5}
\definecolor{codepurple}{rgb}{0.58,0,0.82}
\definecolor{backcolour}{rgb}{0.95,0.95,0.92}

\lstdefinestyle{mystyle}{
    backgroundcolor=\color{backcolour},
    commentstyle=\color{codegreen},
    keywordstyle=\color{magenta},
    numberstyle=\tiny\color{codegray},
    stringstyle=\color{codepurple},
    basicstyle=\ttfamily\footnotesize,
    breakatwhitespace=false,         
    breaklines=true,
    captionpos=b,
    keepspaces=true,
    numbers=left,
    numbersep=5pt,
    showspaces=false,
    showstringspaces=false,
    showtabs=false,
    tabsize=2
}

\begin{document}

% Title page 
\begin{titlepage}
\newpage

\begin{center}
САНКТ-ПЕТЕРБУРГСКИЙ ПОЛИТЕХНИЧЕСКИЙ УНИВЕРСИТЕТ ИМЕНИ ПЕТРА ВЕЛИКОГО \\
\vspace{1em}
\largeВысшая школа прикладной математики и механики \\
\large Кафедра прикладной математики и информатики \\ 
\end{center}

\vspace{8em}


\begin{center}
\textsc{\textbf{Лабораторная работа #1}}\\
\end{center}

\vspace{6em}



\newbox{\lbox}
\savebox{\lbox}{\hbox{}}
\newlength{\maxl}
\setlength{\maxl}{\wd\lbox}
\hfill\parbox{11cm}{
\hspace*{5cm}\hspace*{-5cm}Студент:\hfill\hbox to\maxl{Соломатин Макар Александрович\hfill}\\
\hspace*{5cm}\hspace*{-5cm}Преподаватель:\hfill\hbox to\maxl{Александр Николаевич Баженов}\\
\\
\hspace*{5cm}\hspace*{-5cm}Группа: \hfill\hbox to\maxl{3630102/70201}\\
}


\vspace{\fill}

\begin{center}
Санкт-Петербург \\2020
\end{center}

\end{titlepage}

\newpage

\chapter{Постановка задачи}
\textbf{Задача 1}. Имеется интервальная матрица 
\begin{equation*}
    A_{m,n} = 
	\begin{pmatrix}
	    a_{1,1} & a_{1,2} & \cdots & a_{1,n} \\
	    a_{2,1} & a_{2,2} & \cdots & a_{2,n} \\
	    \vdots  & \vdots  & \ddots & \vdots  \\
	    a_{m,1} & a_{m,2} & \cdots & a_{m,n} 
	\end{pmatrix}
\end{equation*}

\end{document}
