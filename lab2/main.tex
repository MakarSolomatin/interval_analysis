\documentclass[14pt,a4paper,article]{ncc}
\usepackage[a4paper, mag=1000, left=2.5cm, right=1cm, top=2cm, bottom=2cm, headsep=0.7cm, footskip=1cm]{geometry}
\usepackage[utf8]{inputenc}
\usepackage[T2A]{fontenc}
\usepackage[russian]{babel} %[english,russian]
\usepackage{indentfirst}
\usepackage[dvipsnames]{xcolor}
\usepackage[pdftex,unicode=true,colorlinks,filecolor=black,citecolor=black,linkcolor=black]{hyperref}
\usepackage{amsfonts}
\usepackage{amsmath}
\usepackage{listings}
\usepackage{xcolor}
\usepackage{float}

\usepackage{fancyhdr}
\pagestyle{fancy}
\fancyhead[LE,RO]{\thepage}
\fancyfoot{} 

\begin{document}
\begin{titlepage}
    \begin{center}
        САНКТ-ПЕТЕРБУРГСКИЙ ПОЛИТЕХНИЧЕСКИЙ УНИВЕРСИТЕТ ИМЕНИ ПЕТРА ВЕЛИКОГО \\
        \vspace{1em}
        \large Высшая школа прикладной математики и механики \\
        \large Кафедра прикладной математики и информатики \\ 
    \end{center}
    \vspace{8em}
    \begin{center}
        \textsc{\textbf{Лабораторная работа #2}}\\
    \end{center}
    \vspace{14em}
    \newbox{\lbox}
    \savebox{\lbox}{\hbox{Соломатин Макар Александр}}
    \newlength{\maxl}
    \setlength{\maxl}{\wd\lbox}
    \hfill\parbox{11cm}{
    \hspace*{5cm}\hspace*{-5cm}Студент:\hfill\hbox to\maxl{Макар Александрович Соломатин\hfill}\\
    \hspace*{5cm}\hspace*{-5cm}Преподаватель:\hfill\hbox to\maxl{Александр Николаевич Баженов}\\
    \\
    \hspace*{5cm}\hspace*{-5cm}Группа: \hfill\hbox to\maxl{3630102/70201}\\
    }
    \vspace{\fill}
    \begin{center}
        Санкт-Петербург \\2020
    \end{center}
\end{titlepage}

\tableofcontents
\newpage

\section{Постановка задачи}
Использовать функцию $\text{globopt}$ для демонстрации интервальной глобальной оптимизации.
Она возвращает значение глобального экстремума $Z$ и рабочий список $\text{WorkList}$.

$$ \text{function } [Z, WorkList] = \text{globopt0}(X) $$

\subsection{Задача 1}
Построить рабочий список и график сужения интервала для функции Розенбока 4:
$$ f_R = 100 \cdot (x_1^2 - x_2)^2 - (x_1 - 1)^2$$.
Брус, на котором ищется решение $\textbf{X} = [-30, 30] \times [-30, 30]$

\subsection{Задача 2}
Применить метод к функции Матьяса:
$$ f = 0.26(x^2 + y^2) - 0.48x y $$
Исследовать сходимость метода.

\section{Теория}
\subsection{Алгоритм globopt}
Алгоритм для глобальной минимизации функции GlobOpt оперирует с рабочим списком $L$, в котором будут храниться все
брусы, получающиеся в результате дробления исходного бруса области определения на более мелкие подбрусы. Одновременно с самими
подбрусами будем хранить в рабочем списке и нижние оценки областей значений целевой функции по этим подбрусам, так что
элементами списка $L$ будут записи-пары вида
$$L:\; (\textbf{Y}, y)\text{, где} \textbf{Y} \subseteq \textbf{X},\; y = \textbf{f(Y)}$$

Первоначально в рабочий список помещается одна запись
$$(\textbf{X}, \textbf{f(X)})$$
и далее каждый шаг алгоритма состоит в извлечении из этого списка
бруса, который обеспечивает рекордную (т. е. наименьшую) на данный
момент оценку минимума снизу, его дроблении на более мелкие
подбрусы, оценивании на них целевой функции, занесении результатов
обратно в рабочий список.

На $k$-ом шаге алгоритма рабочий список $L$ состоит из $k$ штук
записей-пар вида $(\textbf{Y}^{i,k},y^{i,k}),\; i=1,2\cdots,k$, и для удобства обработки
мы будем считать их упорядоченными по возрастанию значений
второго поля, т. е.
$$L = \{(\textbf{Y}^{1,k}, y^{1,k}),\cdots,(\textbf{Y}^{i,k},y^{i,k})\}$$
где 
$$\textbf{Y}^{i,k} \subseteq \textbf{X}, \;y^{i,k} = \underline{\textbf{f}(\textbf{Y}^{i,k})},\; y^{i,k} \leq y^{j,k}, \;\text{при } i < j$$

Особую роль в исполнении алгоритма играет первая запись списка $L$ ,
которую называют ведущей записью. Соответственно, брус $\textbf{Y}$ будет
называться ведущим брусом, а оценка $y$ — ведущей оценкой для
данного шага алгоритма. Ведущая оценка является наилучшей
текущей оценкой минимума целевой функции снизу, достигнутой
алгоритмом к данному шагу.

\subsection{Функции для оптимизации}
\begin{enumerate}
    \item Функция Розенброка 4:
    $$f_R = 100(x_1^2 -x_2)^2 - (x_1 - 1)^2$$
    График функции имеет тип седло. Минимум функции достигается при значении аргумента $x = (1,1)$ и равен $0$.
    \begin{figure}[h]
        \center{\includegraphics[scale=0.5]{rosenbrok4.png}}
        \caption{Розенброк 4}
        \label{fig:image}
    \end{figure}

    \item Функция сферы:
    $$f_S = x_1^2 + x_2^2$$
    Минимум функции достигается при значении аргумента $x = (0,0)$ и равен $0$.
    \begin{figure}[H]
        \center{\includegraphics[scale=0.25]{sphere.png}}
        \caption{Функция сферы}
        \label{fig:image}
    \end{figure}
\end{enumerate}

\section{Реализация}
Задача решена в среде GNU Octave с использованием пакета $interval$.
Код программы приведен в Приложении 1.

\section{Результаты}
\subsection{Функция Розенброка 4}
Полученные результаты для функции Розенбрка 4 оформлены в виде графиков:
\begin{enumerate}
    \item График \textit{сходимости} - график зависимости расстояния до истинного минимум от номера итерации.
    \item \textit{Траектория центров брусов} - центры ведущих брусов, соединенные прямыми.
    \item \textit{Сужение брусов} - график зависимости диаметра ведущего бруса от номера итерации.
\end{enumerate}

    \begin{figure}[H]
        \center{\includegraphics[scale=0.7]{rosenbrok_conv.png}}
        \caption{Розенброк 4. Сходимость.}
        \label{fig:image}
    \end{figure}
    \begin{figure}[H]
        \center{\includegraphics[width=100mm,scale=0.4]{rosenbrok_traj.png}}
        \caption{Розенброк 4. Траектория цетров брусов.}
        \label{fig:image}
    \end{figure}  
    \begin{figure}[H]
        \center{\includegraphics[scale=0.5]{rosenbrok_contr.png}}
        \caption{Розенброк 4. Диаметры брусов.}
        \label{fig:image}
    \end{figure} 
\subsection{Функция сферы}
Аналогичные графики для функции сферы:
    \begin{figure}[H]
        \center{\includegraphics[scale=0.7]{simple_conf.png}}
        \caption{Функция сферы. Сходимость.}
        \label{fig:image}
    \end{figure}
    \begin{figure}[H]
        \center{\includegraphics[width=100mm,scale=0.4]{simple_traj.png}}
        \caption{Функция сферы. Траектория цетров брусов.}
        \label{fig:image}
    \end{figure}  
    \begin{figure}[H]
        \center{\includegraphics[scale=0.5]{simple_contr.png}}
        \caption{Функция сферы. Диаметры брусов.}
        \label{fig:image}
    \end{figure} 

\section{Вывод}
\subsection{Задача 1}
\subsection{Задача 2}

\section{Приложение}
\begin{enumerate}
    \item Репозиторий с исходным кодом \\ \url{https://github.com/MakarSolomatin/interval_analysis}
\end{enumerate}

\begin{thebibliography}{9}
    \bibitem{lectures} А. Н. Баженов. \textit{Лекции по интервальному анализу}.
    \bibitem{wiki} \\ \url{https://ru.wikipedia.org/wiki/%D0%A2%D0%B5%D1%81%D1%82%D0%BE%D0%B2%D1%8B%D0%B5_%D1%84%D1%83%D0%BD%D0%BA%D1%86%D0%B8%D0%B8_%D0%B4%D0%BB%D1%8F_%D0%BE%D0%BF%D1%82%D0%B8%D0%BC%D0%B8%D0%B7%D0%B0%D1%86%D0%B8%D0%B8}
\end{thebibliography}

\end{document}
